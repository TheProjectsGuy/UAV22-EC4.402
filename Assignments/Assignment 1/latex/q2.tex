% !TeX root = main.tex
\section{Dynamic Thrust Given Propellers}

\paragraph*{UIUC and APC}
The UIUC (University of Illinois at Urbana-Champaign) propeller database can be found at \cite{uiuc-database}.

\noindent
APC is a propeller manufacturer based in California, USA. Propellers are listed \href{https://www.apcprop.com/product-category/multi-copters-drones/}{here}. The APC propellers can be found in UIUC \href{https://m-selig.ae.illinois.edu/props/volume-1/propDB-volume-1.html#APC}{here}. All results presented here are summarized in table \ref{tab:q2-thrust-vals}.

\paragraph*{Reading Method}
Say APC 10x4.7 is given. This means a 10-inch diameter propeller with a pitch of 4.7 inch/revolution. The pitch gives an indication of the amount of twist in propeller blades.

\paragraph*{Equations}
The following equations can be found from \cite{brandt2011propeller,aerotrash-uiuc-article,uiuc-data-equs}.

\begin{align}
    J = \frac{V}{nD} &&
    C_{T} = \frac{T}{\rho n^2 D^4} &&
    C_{P} = \frac{P}{\rho n^3 D^5} &&
    \eta = J \frac{C_T}{C_P}
    \label{eq:q2-base-equs}
\end{align}

The terms in equation \ref{eq:q2-base-equs} are described below

\begin{itemize}
    \item $J$ is called the \emph{Advance ratio}.
    \item $V$ is the \emph{airspeed} (in $m/s$).
    \item $n$ is the rotation speed in $rev/s$. This is also shown as $\Omega$ in the plots.
    \item $D$ is the propeller diameter (in $m$). This is usually interpreted from the name.
    \item $C_T$ is the thrust coefficient.
    \item $T$ is the thrust produced by the propeller (in $N$).
    \item $\rho$ is the air density (in $kg/m^3$).
    \item $C_P$ is the power coefficient.
    \item $P$ is the power (in $W$) which is the product of torque and angular speed. This is the input power.
    \item $\eta$ is the efficiency of the propeller in the particular setting. It is same as the ratio of output power $P_{out} = TV$ to input power $P_{in} = P$, that is $\eta = \sfrac{P_{out}}{P_{in}}$.
\end{itemize}

From equation \ref{eq:q2-base-equs}, the thrust of the propeller can be calculated as

\begin{equation}
    T = C_T \, \rho n^2 D^4
    \label{eq:q2-prop-thrust}
\end{equation}

\subsection{APC 8\texorpdfstring{$\times$}{by}6}

The propeller diameter $D$ is $8$ inches ($0.2032\,m$) and the pitch is $6$ inches per revolution. Assuming the density of air as $1.29\,Kg/m^3$.
Assuming the fight speed to be the air speed, that is $V = 10\,m/s$.

\subsubsection*{4000 RPM}

Here, $n$ is $4000\,RPM$ (or $\sfrac{200}{3}\,rev/s$). First, to calculate $J$, we know $V$, $D$, and $n$.

\begin{equation*}
    J = \frac{V}{nD} = \frac{10\,m/s}{\sfrac{200}{3}\,rev/s \; 0.2032\,m} = \frac{375}{508} = 0.7381
\end{equation*}

The value of $C_T$ can be found from \href{https://m-selig.ae.illinois.edu/props/volume-1/plots/apcsf_8x6_ct.png}{here} (data \href{https://m-selig.ae.illinois.edu/props/volume-1/data/apcsf_8x6_2784rd_4010.txt}{here}). Getting data from line $J = 0.726$. We get $C_T = 0.0496$.

By substituting all this in equation \ref{eq:q2-prop-thrust}, we get the following

\begin{equation*}
    T = C_T \, \rho n^2 D^4 = 0.0496 \times 1.29 \times (\sfrac{200}{3})^2 \times (0.2032)^4 \approx 0.48482\,N
\end{equation*}

At $4000\,RPM$, the dynamic thrust is $0.48482\,N$.

\subsubsection*{5000 RPM}

Here, $n$ is $5000\,RPM$ (or $\sfrac{250}{3}\,rev/s$). Substituting in $J$, we get

\begin{equation*}
    J = \frac{V}{nD} = \frac{10\,m/s}{\sfrac{250}{3}\,rev/s \; 0.2032\,m} = \frac{75}{127} = 0.59055
\end{equation*}

The value of $C_T$ can be found from \href{https://m-selig.ae.illinois.edu/props/volume-1/plots/apcsf_8x6_ct.png}{here} (data \href{https://m-selig.ae.illinois.edu/props/volume-1/data/apcsf_8x6_2786rd_5004.txt}{here}). Getting data from line $J=0.609$. We get $C_T = 0.0723$. Substituting all this in equation \ref{eq:q2-prop-thrust}, we get the following

\begin{equation*}
    T = C_T \, \rho n^2 D^4 = 0.0723 \times 1.29 \times (\sfrac{250}{3})^2 \times (0.2032)^4 \approx 1.10423\,N
\end{equation*}

At $5000\,RPM$, the dynamic thrust is $1.10423\,N$.

\subsubsection*{6000 RPM}

Here, $n$ is $6000\,RPM$ (or $100\,rev/s$). Substituting in $J$, we get

\begin{equation*}
    J = \frac{V}{nD} = \frac{10\,m/s}{100\,rev/s \; 0.2032\,m} = \frac{125}{254} = 0.492125
\end{equation*}

The value of $C_T$ can be found from \href{https://m-selig.ae.illinois.edu/props/volume-1/plots/apcsf_8x6_ct.png}{here} (data \href{https://m-selig.ae.illinois.edu/props/volume-1/data/apcsf_8x6_2788rd_6009.txt}{here}). Getting data from line $J = 0.507$. We get $C_T = 0.0970$. Substituting all this in equation \ref{eq:q2-prop-thrust}, we get the following

\begin{equation*}
    T = C_T \, \rho n^2 D^4 = 0.0970 \times 1.29 \times (100)^2 \times (0.2032)^4 \approx 2.133321\,N
\end{equation*}

At $6000\,RPM$, the dynamic thrust is $2.133321\,N$.

\subsection{APC 9\texorpdfstring{$\times$}{by}6}

The propeller diameter $D$ is $9$ inches ($0.2286\,m$) and the pitch is $6$ inches per revolution. Assuming the density of air as $1.29\,Kg/m^3$.
Assuming the fight speed to be the air speed, that is $V = 10\,m/s$.

\subsubsection*{4000 RPM}

Here, $n$ is $4000\,RPM$ (or $\sfrac{200}{3}\,rev/s$). Substituting in $J$, we get

\begin{equation*}
    J = \frac{V}{nD} = \frac{10\,m/s}{\sfrac{200}{3}\,rev/s \; 0.2286\,m} = \frac{250}{381} = 0.65616
\end{equation*}

The value of $C_T$ can be found from \href{https://m-selig.ae.illinois.edu/props/volume-1/plots/apcsf_9x6_ct.png}{here} (data \href{https://m-selig.ae.illinois.edu/props/volume-1/data/apcsf_9x6_kt0980_4016.txt}{here}). Getting data from line $J=0.648$. We get $C_T=0.0515$. Substituting all this in equation \ref{eq:q2-prop-thrust}, we get the following

\begin{equation*}
    T = C_T \, \rho n^2 D^4 = 0.0515 \times 1.29 \times (\sfrac{200}{3})^2 \times (0.2286)^4 \approx 0.80634\,N
\end{equation*}

At $4000\,RPM$, the dynamic thrust is $0.80634\,N$.

\subsubsection*{5000 RPM}

Here, $n$ is $5000\,RPM$ (or $\sfrac{250}{3}\,rev/s$). Substituting in $J$, we get

\begin{equation*}
    J = \frac{V}{nD} = \frac{10\,m/s}{\sfrac{250}{3}\,rev/s \; 0.2286\,m} = \frac{200}{381} = 0.524934
\end{equation*}

The value of $C_T$ can be found from \href{https://m-selig.ae.illinois.edu/props/volume-1/plots/apcsf_9x6_ct.png}{here} (data \href{https://m-selig.ae.illinois.edu/props/volume-1/data/apcsf_9x6_kt0982_5006.txt}{here}). Getting data from line $J = 0.541$. We get $C_T = 0.0773$. Substituting all this in equation \ref{eq:q2-prop-thrust}, we get the following

\begin{equation*}
    T = C_T \, \rho n^2 D^4 = 0.0773 \times 1.29 \times (\sfrac{250}{3})^2 \times (0.2286)^4 \approx 1.89108\,N
\end{equation*}

At $5000\,RPM$, the dynamic thrust is $1.89108\,N$.

\subsubsection*{6000 RPM}

Here, $n$ is $6000\,RPM$ (or $100\,rev/s$). Substituting in $J$, we get

\begin{equation*}
    J = \frac{V}{nD} = \frac{10\,m/s}{100\,rev/s \; 0.2286\,m} = \frac{500}{1143} = 0.437445
\end{equation*}

The value of $C_T$ can be found from \href{https://m-selig.ae.illinois.edu/props/volume-1/plots/apcsf_9x6_ct.png}{here} (data \href{https://m-selig.ae.illinois.edu/props/volume-1/data/apcsf_9x6_kt0983_6017.txt}{here}). Getting data from line $J = 0.430$. We get $C_T = 0.1039$. Substituting all this in equation \ref{eq:q2-prop-thrust}, we get the following

\begin{equation*}
    T = C_T \, \rho n^2 D^4 = 0.1039 \times 1.29 \times (100)^2 \times (0.2286)^4 \approx 3.66024\,N
\end{equation*}

At $6000\,RPM$, the dynamic thrust is $3.66024\,N$.

\subsection{APC 10\texorpdfstring{$\times$}{by}7}

The propeller diameter $D$ is $10$ inches ($0.254\,m$) and the pitch is $7$ inches per revolution. Assuming the density of air as $1.29\,Kg/m^3$.
Assuming the fight speed to be the air speed, that is $V = 10\,m/s$.

\subsubsection*{4000 RPM}

Here, $n$ is $4000\,RPM$ (or $\sfrac{200}{3}\,rev/s$). Substituting in $J$, we get

\begin{equation*}
    J = \frac{V}{nD} = \frac{10\,m/s}{\sfrac{200}{3}\,rev/s \; 0.254\,m} = \frac{75}{127} = 0.59055
\end{equation*}

The value of $C_T$ can be found from \href{https://m-selig.ae.illinois.edu/props/volume-1/plots/apcsf_10x7_ct.png}{here} (data \href{https://m-selig.ae.illinois.edu/props/volume-1/data/apcsf_10x7_kt0830_3999.txt}{here}). Getting data from line $J = 0.606$. We get $C_T = 0.0582$. Substituting all this in equation \ref{eq:q2-prop-thrust}, we get the following

\begin{equation*}
    T = C_T \, \rho n^2 D^4 = 0.0582 \times 1.29 \times (\sfrac{200}{3})^2 \times (0.254)^4 \approx 1.38888\,N
\end{equation*}

At $4000\,RPM$, the dynamic thrust is $1.38888\,N$.

\subsubsection*{5000 RPM}

Here, $n$ is $5000\,RPM$ (or $\sfrac{250}{3}\,rev/s$). Substituting in $J$, we get

\begin{equation*}
    J = \frac{V}{nD} = \frac{10\,m/s}{\sfrac{250}{3}\,rev/s \; 0.254\,m} = \frac{60}{127} = 0.47244
\end{equation*}

The value of $C_T$ can be found from \href{https://m-selig.ae.illinois.edu/props/volume-1/plots/apcsf_10x7_ct.png}{here} (data \href{https://m-selig.ae.illinois.edu/props/volume-1/data/apcsf_10x7_kt0831_5003.txt}{here}). Getting data from line $J = 0.482$. We get $C_T = 0.0872$. Substituting all this in equation \ref{eq:q2-prop-thrust}, we get the following

\begin{equation*}
    T = C_T \, \rho n^2 D^4 = 0.0872 \times 1.29 \times (\sfrac{250}{3})^2 \times (0.254)^4 \approx 3.25146\,N
\end{equation*}

At $5000\,RPM$, the dynamic thrust is $3.25146\,N$.

\subsubsection*{6000 RPM}

Here, $n$ is $6000\,RPM$ (or $100\,rev/s$). Substituting in $J$, we get

\begin{equation*}
    J = \frac{V}{nD} = \frac{10\,m/s}{100\,rev/s \; 0.254\,m} = \frac{50}{127} = 0.39370
\end{equation*}

The value of $C_T$ can be found from \href{https://m-selig.ae.illinois.edu/props/volume-1/plots/apcsf_10x7_ct.png}{here} (data \href{https://m-selig.ae.illinois.edu/props/volume-1/data/apcsf_10x7_kt0833_6006.txt}{here}). Getting data from line $J = 0.382$. We get $C_T = 0.1138$. Substituting all this in equation \ref{eq:q2-prop-thrust}, we get the following

\begin{equation*}
    T = C_T \, \rho n^2 D^4 = 0.1138 \times 1.29 \times (100)^2 \times (0.254)^4 \approx 6.11036\,N
\end{equation*}

At $6000\,RPM$, the dynamic thrust is $6.11036\,N$.

\begin{table}[ht]
    \centering
    \begin{tabular}{|l||*{5}{c|}} \hline
        \backslashbox{Model}{RPM} & 4000 & 5000 & 6000 \\ \hline
        APC 8$\times$6 & $0.48482\,N$ & $1.10423\,N$ & $2.133321\,N$ \\\hline
        APC 9$\times$6 & $0.80634\,N$ & $1.89108\,N$ & $3.66024\,N$ \\\hline
        APC 10$\times$7 & $1.38888\,N$ & $3.25146\,N$ & $6.11036\,N$ \\\hline
    \end{tabular}
    \caption{Table of thrust values for different propellers}
    \small
        Dynamic thrust of different APC propellers with $10\,m/s$ flight speed.
    \label{tab:q2-thrust-vals}
\end{table}
