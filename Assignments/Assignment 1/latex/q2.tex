% !TeX root = main.tex
\section{Dynamic Thrust Given Propellers}

The UIUC (University of Illinois at Urbana-Champaign) propeller database can be found at \cite{uiuc-database}.

\noindent
APC is a propeller manufacturer based in California, USA. Propellers are listed \href{https://www.apcprop.com/product-category/multi-copters-drones/}{here}. The APC propellers can be found in UIUC \href{https://m-selig.ae.illinois.edu/props/volume-1/propDB-volume-1.html#APC}{here}.

\paragraph*{Reading Method}
Say APC 10x4.7 is given. This means a 10-inch diameter propeller with a pitch of 4.7 inch/revolution. The pitch gives an indication of the amount of twist in propeller blades.

\paragraph*{Equations}
The following equations can be found from \cite{brandt2011propeller,aerotrash-uiuc-article,uiuc-data-equs}.

\begin{align}
    J = \frac{V}{nD} &&
    C_{T} = \frac{T}{\rho n^2 D^4} &&
    C_{P} = \frac{P}{\rho n^3 D^5} &&
    \eta = J \frac{C_T}{C_P}
    \label{eq:q2-base-equs}
\end{align}

The terms in equation \ref{eq:q2-base-equs} are described below

\begin{itemize}
    \item $J$ is called the \emph{Advance ratio}.
    \item $V$ is the \emph{airspeed} (in $m/s$).
    \item $n$ is the rotation speed in $rev/s$. This is also shown as $\Omega$ in the plots.
    \item $D$ is the propeller diameter (in $m$). This is usually interpreted from the name.
    \item $C_T$ is the thrust coefficient.
    \item $T$ is the thrust produced by the propeller (in $N$).
    \item $\rho$ is the air density (in $kg/m^3$).
    \item $C_P$ is the power coefficient.
    \item $P$ is the power (in $W$) which is the product of torque and angular speed. This is the input power.
    \item $\eta$ is the efficiency of the propeller in the particular setting. It is same as the ratio of output power $P_{out} = TV$ to input power $P_{in} = P$, that is $\eta = \sfrac{P_{out}}{P_{in}}$.
\end{itemize}

From equation \ref{eq:q2-base-equs}, the thrust of the propeller can be calculated as

\begin{equation}
    T = C_T \, \rho n^2 D^4
    \label{eq:q2-prop-thrust}
\end{equation}

\subsection{APC 8\texorpdfstring{$\times$}{by}6}

The propeller diameter $D$ is $8$ inches ($0.2032\,m$) and the pitch is $6$ inches per revolution. Assuming the density of air as $1.29\,Kg/m^3$.
Assuming the fight speed to be the air speed, that is $V = 10\,m/s$

\subsubsection*{4000 RPM}

Here, $n$ is $4000\,RPM$ (or $\sfrac{200}{3}\,rev/s$). First, to calculate $J$, we know $V$, $D$, and $n$.

\begin{equation*}
    J = \frac{V}{nD} = \frac{10\,m/s}{\sfrac{200}{3}\,rev/s \; 0.2032\,m} = \frac{375}{508} = 0.7381
\end{equation*}

The value of $C_T$ can be found from \href{https://m-selig.ae.illinois.edu/props/volume-1/plots/apcsf_8x6_ct.png}{here} (data \href{https://m-selig.ae.illinois.edu/props/volume-1/data/apcsf_8x6_2784rd_4010.txt}{here}). Getting data from line $J = 0.726$. We get $C_T = 0.0496$.

By substituting all this in equation \ref{eq:q2-prop-thrust}, we get the following

\begin{align*}
    T = C_T \, \rho n^2 D^4
\end{align*}
